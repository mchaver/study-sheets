\documentclass[12pt,a4j, landscape, dvipdfmx]{utarticle}

\usepackage{color} % not required for this example...
\usepackage{hhline}
\pagenumbering{gobble}

\addtolength{\textheight}{.6in}
\addtolength{\topmargin}{-1.5cm}

% height is 21 cm
% width is 29.7 cm

\def\hcell(#1){
  \vspace*{-.1cm}
  \rule{0pt}{0ex} \hspace*{1.2cm}  {\LARGE #1}
  \vspace*{.2cm}
}

\def\xcell(#1){
  \rule{0pt}{3ex} {\LARGE #1}
  \vspace*{.2cm}
}

\def\ncell(#1){
  \rule{0pt}{3ex} {\Large #1}
}

\def\scell(#1){
  \rule{0pt}{3ex} {\small #1}
}

\def\tnewline{
  \tabularnewline 
  \hhline{|:=::=::=:|}
}

\begin{document}

% for some reason the size of the page forces it to be printed on the second 
% page. We can use a minipage to force it onto the first page.

\noindent\makebox[\textwidth][c]{
\begin{minipage}[t][0pt]{\linewidth}

% in table height and width are flipped

\noindent
\begin{tabular}{||p{5.5cm} || p{5.5cm} || p{5.5cm} ||}
\hhline{|t:=:t:=:t:=:t|}

\hcell(丁寧語) & \hcell(尊敬語) & \hcell(謙譲語)
\tnewline

\xcell(見ます) & \xcell(ご覧になる) & \xcell(拝見する)
\tnewline

\xcell(会います) & \xcell(お会いになる) & \xcell(お目にかかる)
\tnewline

\xcell(あります) & \xcell(いらっしゃる) & \xcell(ございます)
\tnewline

\xcell(あります) & \xcell(おいでになる) & \xcell(ございます)
\tnewline

\xcell(います) & \xcell(いらっしゃる) & \xcell(おる)
\tnewline

\xcell(います) & \xcell(おいでになる) & \xcell(おる)
\tnewline

\ncell(行きます / 来ます) & \scell(いらっしゃる / おいでになる) & \xcell(伺う / 参る)
\tnewline

\xcell(知ります) & \xcell(ご存じ) & \xcell(存じ上げる)
\tnewline

\xcell(食べます) & \xcell(召し上がる) & \ncell(いただく / 頂戴する)
\tnewline

\xcell(飲みます) & \xcell(召し上がる) & \ncell(いただく / 頂戴する)
\tnewline

\xcell(もらいます) & \xcell() & \xcell(いただく)
\tnewline

\xcell(あげます) & \xcell() & \xcell(差しあげる)
\tnewline

\xcell(くれます) & \xcell(くださる) & \xcell()
\tnewline

\xcell(言います) & \xcell(おっしゃる) & \ncell(申し上げる / 申す)
\tnewline

\xcell(着ます) & \xcell(お召しになる) & \xcell()
\tnewline


\xcell(寝ます) & \xcell(お休みになる) & \xcell()
\tnewline

\xcell(死にます) & \ncell(お亡くなりになる) & \xcell()
\tnewline

\xcell(聞きます) & \xcell() & \xcell(伺う)
\tnewline

\xcell(訪ねます) & \xcell() & \xcell(伺う)
\\
\hhline{|b:=:b:=:b:=:b|}

\end{tabular} 

\end{minipage}
}
\end{document}
